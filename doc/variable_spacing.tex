\documentclass[12pt,final]{article}

\usepackage{amsmath}
\usepackage{amssymb}

\begin{document}

\title{Determining Box-Joint Finger Widths of Variable Size}
\date{September 10, 2015}
\author{Robert B. Lowrie}
\maketitle

\section{Introduction}

In this document, we outline methods for computing box-joint finger widths
with the following properties:
\begin{enumerate}
\item The center finger is the widest and is centered on the midpoint
of the joint.
\item Each finger away from the center finger is a width-$d$ less than
  its adjoining finger towards the centerline.
\end{enumerate}
Note that the finger widths are such that the pattern is symmetric
about the centerline of the joint.  Figure [] shows the example
pattern.

Be warned, this document is written using a lot of mathematics notation and
contains a lot of algebra.  I've tried to simplify the notation as much as I
can.  If you're uncomfortable with the math, that's fine.  You're probably
also a much better woodworker than I am.

The nomenclature used in this document is as follows:
\begin{itemize}
  \item $B$ = The router bit (or dado) width.  This may also be viewed as the
    minimum width of a interior cut.  I should note here that this document
    can be used regardless of your preferred unit of measure; it holds for the
    English system just as well as for metric.
  \item $W$ = total width of the joint.  We assume in this document that $W$
    is an exact multiple of $B$.
  \item $N$ = total number of fingers, both on one board and the
    adjoining board.  Because of the joint symmetry about the center,
    we know that $N$ is an odd number, with $(N-1)/2$
    fingers on each side of the center finger.
  \item $M = (N-1)/2$.  On the edge with the centered cut, this is the number
    of cuts to one side.
  \item $d$ = the change in width from a finger to its adjoining finger,
    from the adjoining board.  $d =
    0$ gives a traditional equally-spaced, box joint (with a caveat to be
    explained below).
  \item $f_i$ = width of finger $i$, where $i = 0, 1, 2, \ldots, M$.
    The indexing is defined so that the center finger has width $f_0$,
    it has two adjoining fingers of width $f_1$, and we strive to have $d = f_{i} -
    f_{i+1}$.
\end{itemize}

\section{Equally-spaced case}

As a special case, we want to be able to handle the equally-spaced case; that
is, when $d = 0$.  Let
\begin{equation}
  W_b = \frac{W}{B}\,.
\end{equation}
Note that with the assumptions above, $W_b$ is an exact integer.  There then are
two cases:
\begin{enumerate}
\item $W_b$ is odd.  Then $N = W_b$ and all fingers are width $B$.
\item $W_b$ is even.  Then $N = W_b+1$ and all fingers are width $B$, except for
  two fingers at the edge of the joint, which are width $B / 2$.  See Fig. []
  for an example.
\end{enumerate}
For the second case, another possibility is to offset the joints over by $B/2$.
Then the pattern would no longer be symmetric about the center of the edge.
It would be anti-symmetric.  That's fine for standard box joints, 
but when we consider $d > 0$, we feel we need to maintain symmetry about the
centerline. We'll discuss this issue more in the next below.

\section{Derivation of Finger Widths for the General Case}

The sum of the finger widths must equal the total width of the joint, so that
\begin{align}
  W &= f_0 + 2 (f_1 + f_2 + \ldots + f_M)\,,\\
    &= f_0 + 2 \sum_{i = 1}^{M} f_i\,.
\end{align}
To within the resolution of our measuring (say 1/32 inch or 1 mm), we want $d =
f_i - f_{i+1}$, which gives
\begin{equation}
  \label{eq:fi}
  f_i = f_0 - i d\,.
\end{equation}
The only exception is for even-$W_b$, in which case for $i = M$ (the fingers at
the edge) we set $f_{M} = (f_{M-1} - d)/ 2$, which can be written as
\begin{equation}
  f_M = \frac{f_0 - M d}{2}\,.
\end{equation}
For $W_b$ either even or odd, we can write
\begin{equation}
  \label{eq:fm}
  f_M = \frac{f_0 - M d}{1 + \alpha}\,,
\end{equation}
where $\alpha = 0$ for odd-$W_b$, and $\alpha = 1$ when even-$W_b$, so
\begin{equation}
     \alpha = \text{mod}(W_b+ 1, 2)\,.
\end{equation}

The width may then be written as
\begin{equation}
  W = f_0 + 2 \sum_{i = 1}^{M} (f_0 - i d) - \alpha (f_0 - M d)\,.
\end{equation}
Next use the relations
\begin{equation}
   2 \sum_{i = 1}^{M} 1 = N - 1, \qquad
   2 \sum_{i = 1}^{M} i = M (M+1),
\end{equation}
so that we have
\begin{equation}
 \label{eq:form1}
  W = N f_0 - d M (M+1) - \alpha (f_0 - M d)\,.
\end{equation}
Note that if $d = 0$, then the finger widths are all the same and we
get back the familiar relationship $W = W_b f_0$.

Alternative forms for (\ref{eq:form1}) are also useful.  Let
\begin{itemize}
\item $C = f_0$, the center finger width.
\item $E = (\alpha + 1) f_M$, the finger width at the edge of the joint for
  odd-$W_b$, or twice that for even-$W_b$.
\end{itemize}
We can solve eq. (\ref{eq:fm}) for $d$ to obtain
\begin{equation}
  d = \frac{C - E}{M}\,.
\end{equation}
Eq. (\ref{eq:form1}) may then be written as
\begin{equation}
  \label{eq:dimformE}
  W = M C + (M + 1 - \alpha) E\,.
\end{equation}
Therefore,
\begin{equation}
  \label{eq:dimE}
  E = \frac{W - M C}{M + 1 - \alpha}\,.
\end{equation}

Another option is to ensure that the minimum interior finger, $f_{M-1}$, is
the bit (or dado) width $B$.  Then (\ref{eq:fi}) gives
\begin{equation}
  d = \frac{C - B}{M-1}\,,
\end{equation}
and
\begin{equation}
  \label{eq:dimformB}
  W = \frac{1}{M-1} \left[C(M^2 - 2 M -1) + B M (1+M) + \alpha (C - B M)\right]
\end{equation}
There are several alternative forms of these relations, discussed below.

\subsection{Normalized by $C$}

We can divide eq. (\ref{eq:dimformE}) by $C$ to give
\begin{equation}
  \label{eq:wc}
  W_c = M + (M + 1) E_c\,,
\end{equation}
where $W_c = W/C$ and $E_c = E / C$.  We require that $0 < E_c \le 1$,
which gives
\begin{equation}
  M < W_c \le N\,.
\end{equation}
A typical case is to specify $E_c$ and $N$ and compute $W_c$. The
normalized eq. (\ref{eq:fi}) becomes
\begin{equation}
  \label{eq:fic}
  \frac{f_i}{C} = 1 - i \frac{1 - E_c}{M}\,.
\end{equation}

A choice I used for a project is $N = 11$ ($M=5$) and $E_c = 0.25$, so that
(\ref{eq:wc},\ref{eq:fic}) reduce to
\begin{equation}
  W_c = 6.5\, \qquad
  \frac{f_i}{C} = 1 - 0.15 i\,.
\end{equation}
\begin{center}
\begin{tabular}{l|l}
  $i$ & $f_i/C$ \\
  \hline
  1 & 0.85 \\
  2 & 0.70 \\
  3 & 0.55 \\
  4 & 0.40 \\
  5 & 0.25 \\
  \hline
\end{tabular}
\end{center}

\subsection{Constrained by the bit width}

In this section, we consider the case where we want to ensure that the
narrowest slot is the bit width, $B$.
Let $C_b = C / B$ and $E_b = E / B$.  Equations
(\ref{eq:dimformB},\ref{eq:fi},\ref{eq:fm}) become
\begin{equation}
  \label{eq:wb}
  W_b = \frac{1}{M-1} \left[C_b(M^2 - 2 M -1) + M (1+M) + 
    \alpha (C_b - B)\right]\,,
\end{equation}
\begin{equation}
  \label{eq:fib}
  \frac{f_i}{B} =  \beta_i \left(C_b - i \frac{C_b - 1}{M-1}\right)\,,
\end{equation}
where
\begin{equation}
  \beta_i = 
  \begin{cases}
      1 & 0 \le i < M\,,\\
      \frac{1}{1 + \alpha} & i = M\,.
  \end{cases}
\end{equation}
In addition, (\ref{eq:dimE}) becomes
\begin{equation}
  \label{eq:eb}
  E_b = \frac{W_b - M C_b}{M + 1 - \alpha}\,.
\end{equation}

Now, the cut $i=M$ is at the edge of the board, which is allowed to be less
than the bit width, because we allow the bit to extend over the ends of the
joint.  But the $i=M$ cut must be non-negative, which requires
\begin{equation}
  C_b - M \frac{C_b - 1}{M-1} \ge 0\,,
\end{equation}
or
\begin{equation}
  C_b \le M\,.
\end{equation}

Next, we may solve (\ref{eq:wb}) for $C_b$ to obtain
\begin{equation}
  \label{eq:cb}
  C_b = \frac{(M-1)W_b - (M+1-\alpha)M}{M^2 - 2M - 1 + \alpha}\,.
\end{equation}
Now if we substitute this relation into (\ref{eq:eb}), we obtain
\begin{equation}
  E_b = \frac{M^2 - W_b}{(1 + \alpha)(M^2 - 2M - 1 + \alpha)}\,.
\end{equation}
We must have $E_b > 0$, which is satisfied if
\begin{equation}
  M > \max(3 - \alpha, \sqrt{W_b})\,.
\end{equation}
Equally-spaced fingers are when $W_b = N = 2M + 1 - \alpha$, so combined with
the constraint above, we require that
\begin{gather}
  M_{\min} \le M \le M_{\max}\,\\
\intertext{where}
M_{\min} = \max(3 - \alpha, \sqrt{W_b})\,, \qquad
  M_{\max} = \frac{W_b - 1 + \alpha}{2}\,.
\end{gather}
Smaller values of $M$ (but no smaller than $M_{\min}$) will increase 
the difference in width of the fingers.  Of coarse $M$ must be an integer, so
we compute the constraints as
\begin{equation}
\label{eq:mConstraints}
M_{\min} = \max\left(3-\alpha, \text{ceil}\left(\sqrt{W_b}\right)\right)\,, \quad
  M_{\max} = \text{floor}\left(\frac{W_b - 1 + \alpha}{2}\right)\,.
\end{equation}

\section{Summary}

So let's summarize how we can generate variable-width fingers:
\begin{enumerate}
\item Say you're given an overall joint-width $W$ and bit (or dado) width $B$.
\item Compute $W_b = W / B$.
\item Set $\alpha = 1$ for even-$W_b$ and $\alpha = 0$ for odd-$W_b$.
\item Compute $M_{\min}$ and $M_{\max}$ from eq. (\ref{eq:mConstraints}).
\item Vary $M$ (must be an integer) between $M_{\min}$ and $M_{\max}$
  to generate different finger patterns.  When $M=M_{\max}$, we get a standard
  box joint, with each finger the same width.  When $M=M_{\min}$, we get the
  widest variation in finger sizes.
\item For a given choice of $M$, we can compute $C_b$ from (\ref{eq:cb})
\item Then $C = f_0 = B C_b$.  This is the finger width at the center of the
  joint.
\item For each $i$, $i = 1$ to $i = M$, the finger-widths $f_i$ away from the
  center of the joint can be computed from (\ref{eq:fib}).  
\item One issue is accuracy.  You often get a number such as $f_2 = 1.0438475$
  inches.  You obviously can't measure to that sort of accuracy.
\end{enumerate}

\end{document}
